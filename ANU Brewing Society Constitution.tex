\documentclass{article}

\usepackage[shortlabels]{enumitem}

\title{ANU Brewing Society Constitution}
\author{}
\date{}

\begin{document}

\maketitle

\section{Aims and Objectives of the Club}

\begin{enumerate}
    \item[1.1] We hold these truths to be self-evident that all beer styles are created equal. But not all beers are created equal, we are founded in opposition to industrial pale lager.
    \item[1.2] The club, whilst inspired by the homebrewing of beer should not be limited to the brewing of alcoholic beverages. In the interest of promoting membership with non- drinkers, efforts to brew non-alcoholic beverages should be made. (i.e non-alcoholic ginger beer, kvass or bouza).
    \item[1.3] We shall not discriminate against people on the basis of race, faith, sexuality, gender or beverage preference.
\end{enumerate}

\section{Operation of the Club}

\begin{enumerate}
    \item[2.1] Another truth that is very self-evident is that we will mostly be plagiarising our constitution from the “Affiliating Clubs” document.
    \item[2.2] The Club is affiliated to the Clubs Council of the ANU Students’ Association and anything in this Constitution which is inconsistent with the ANU Students’ Association Clubs Regulations and Policies is null and void to the extent of that inconsistency.
    \item[2.3] The club must fulfil its obligations under the Association’s Clubs Regulations.
    \item[2.4] The club can be dissolved by decision of the current membership at a general meeting, when the club ceases to operate due to no members being elected to the executive/trustee positions, the club ceasing to function through natural attrition, or by a resolution by the Council. 
    \item[2.5] In the case of the club being dissolved any excess assets, property, funds or money remaining after all debts and liabilities are paid shall not be given to members but shall be given or transferred to the ANU Students’ Association.
    \item[2.6] The assets and income of the Club/Society shall be used only for the promotion of the Club/Society’s objectives and no portion may be paid or transferred directly or indirectly to members of the club/society except as:
    \begin{enumerate}[a.]
        \item bona-fide remuneration for services rendered by the members to the Club/Society,
        \item repayment of expenses incurred on behalf of the Club/Society,
        \item interest at a rate not exceeding interest at the rate for the time being which is or would be charged by the clubs/society’s bankers for money lent to the club/society, and
        \item bona-fide rent for premises let to the club/society.
    \end{enumerate}
    \item[2.7] Any equipment which is owned by the club should be stored in the care of a member of the executive or in an area controlled by the brewing society and equipment should always be used with either the supervision or permission of a member of the executive.
    \item[2.8] The club cannot take monetary loans from individuals or organisations. 
\end{enumerate}

\section{Disputes}

\begin{enumerate}
    \item[3.1] In the event of a dispute, the Chief Braumeister, Brewmaster, Alewife or Brewster who holds the office of President has a duty to make it known to the parties to the dispute that an appeal can be brought to the Clubs Council for any decision made by a club’s executive team.
    \item[3.2] The Trustees of a club or society have a legal duty to ensure that that clubs funds are being used in the best interest of the club or society, which is to be viewed from the perspective of the club or society as an independent entity and not from the perspective of members of that club or society.
    \item[3.3] Any member of the executive has a constitutional duty to report to the executive team if a member of the club has served alcohol made by the club to a minor or sold alcohol made by the club. If they fail to do this, they may face disciplinary action.
    \item[3.4] If any member of the club serves club made alcohol to a minor or sells said alcohol they are subject to disciplinary action not limited to expulsion from the club and reporting to relevant authorities.
\end{enumerate}

\section{Elections}

\begin{enumerate}
    \item[4.1] The President of the club is elected by the members of the club at the first general meeting of each semester. The date of this first meeting and election is set by a vote in the last meeting of the previous semester. Members of the executive will not be involved in this vote.
    \item[4.2] The President-elect after their ascension to the office of the President shall not be referred to as the President and must select a regnal title. The options are Chief Braumeister, Chief Brewmaster, Chief Alewife, Chief Brewer or Chief Brewster, these titles may be used regardless of gender or lack thereof.
    \item[4.3] The Treasurer, Secretary and Vice-Chief Brewer (Vice president) will also be elected on the first general meeting of each semester. These must be separate people.
    \item[4.4] If an election is to be held at a General meeting there should be at least one week of notice for the meeting.
    \item[4.5] To be considered for election to the executive a member must have been involved with at least two brews with the club, there is an exception for the first year of club elections.
    \item[4.6] At least one member of the executive should have a valid Australian RSA.
    \item[4.7] If a member of the executive resigns or is rendered unable to perform their duties another member may be co-opted into the vacant position, until the next general meeting. At which point a special election should be called to fill the position.
    \item[4.8] For a general meeting to occur at least two members of the executive should be there and there must be a quorum. One of those executive members should be the Chief Brewer or Vice-Chief Brewer.
    \begin{enumerate}[a.]
        \item Any member of the executive can be removed by a majority vote at a General Meeting.
        \item Members of the executive are bound to follow any decision made by the club relating to the operation of their duties
    \end{enumerate}
    \item[4.9] A quorum has three times as many members as the full executive.
\end{enumerate}

\section{Membership}

\begin{enumerate}
    \item[5.1] Full membership is open to all members of ANUSA and PARSA.
    \begin{enumerate}[a.]
        \item That being said no member under 18 may brew alcoholic drinks with the club, though they are welcome to brew non-alcoholic drinks. This is a legal requirement and not a digressionary limitation by the club.
        \item The membership fee will not exceed \$20.
        \item Membership fees will be paid per semester with the option of annual payment.
    \end{enumerate}
    \item[5.2] All memberships will be valid to the end of the calendar year during which the membership was paid.
    \item[5.3] Any members wishing to continue to participate in the club will need to repay a membership fee each calendar year.
\end{enumerate}

\section{Constitution}
\begin{enumerate}
    \item[6.1] No amendments to this constitution shall be validly enacted except where the Clubs Council Executive is given notice by email of the amendment at least five (5) days prior to the moving of the amendment. The email shall request an interpretation as to whether the proposed amendment will affect the continued eligibility of the club for reaffiliation.
    \item[6.2] Any interpretation of the constitution or the resolution of any dispute under it may be appealed to the Association Clubs Council Secretary. The decision of the Clubs Council Secretary may be appealed by the means set out in the constitution and regulations of the Association.
\end{enumerate}

\end{document}